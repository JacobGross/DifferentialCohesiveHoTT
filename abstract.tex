\documentclass{article}
%\usepackage[lmargin=2.5cm,rmargin=2.5cm,tmargin=2.5cm,bmargin=2.5cm]{geometry}
\usepackage{fullpage}
\usepackage{amsmath}
\usepackage{amssymb}
\usepackage{amsthm}
\usepackage{phonetic} % for esh
\usepackage{enumerate}

\usepackage{tikz}
\usetikzlibrary{arrows}

\usepackage{authblk} % For authors' affiliations
\makeatletter
\renewcommand\AB@affilsepx{, \protect\Affilfont}
\makeatother


\usepackage[hidelinks]{hyperref}
\usepackage{natbib}

\usepackage[capitalize,noabbrev]{cleveref} % must be loaded after hyperref

\usepackage{xspace}

\usepackage{wrapfig}

% Modalities
\newcommand{\Red}{\Re}
\newcommand{\Cored}{\Im}
\newcommand{\Wat}{\&}
\newcommand{\shape}{\ensuremath{\mathord{\raisebox{0.5pt}{\text{\rm\esh}}}}}
\newcommand{\submodality}{\rotatebox[origin=c]{90}{$\subset$}}


% Latin  Abbr
\newcommand{\etal}{\emph{et al.}\xspace}
\newcommand{\eg}{\emph{e.g.,}\xspace}
\newcommand{\ie}{\emph{i.e.,}\xspace}
\newcommand{\etc}{\emph{etc.}\xspace}

%%%%%%%%%%%%%%
%% Comments %%
%%%%%%%%%%%%%%
\usepackage{draft}

\newnote[Jacob]{JAG}{blue}
\newnote[Max]{MN}{blue}
\newnote[Jennifer]{JP}{blue}
\newnote[Mitchell]{MMR}{blue}
\newnote[Felix]{FW}{blue}
\newnote[Dan]{DL}{blue}
\newnote[Mike]{MS}{blue}

\newnote{fixme}{red}
\newnote{note}{red}
\newnote{todo}{red}
\newnote[Cite:]{tocite}{red}




% use \draftfalse to turn off comments
% use \drafttrue to turn on comments (default)
%\draftfalse

\title{Differential Cohesive Type Theory (Extended Abstract)\thanks{This
    material is based upon work supported by the National Science Foundation
    under Grant Number DMS 1321794.}}
\author{Jacob A.\,Gross}
\affil[1]{University of Pittsburgh}
\author[2]{Daniel R.\,Licata}
\affil[2]{Wesleyan University}
\author[3]{Max S.\, New}
\affil[3]{Northeastern University}
\author[4]{Jennifer Paykin}
\affil[4]{University of Pennsylvania}
\author[2]{Mitchell Riley}
\author[5]{Michael Shulman}
\affil[5]{University of San Diego}
\author[6]{Felix Wellen}
\affil[6]{Karlsruhe Institute of Technology}
\date{}

%\author{Jacob A.\,Gross \and Max S.\, New \and Jennifer Paykin \and Mitchell Riley
%  \and Felix Wellen \and Daniel R.\,Licata \and Michael Shulman}

\begin{document}
\maketitle

\JP{How to use comments: use your initials as a command to make a comment
  terminated by your name, \eg \texttt{\textbackslash JP\{my comments\}}. May
  also use \texttt{fixme}, \texttt{note}, \texttt{todo}, and \texttt{tocite}.
  You can change your keyword, color, etc in the preamble. To remove all
  comments, \eg when submitting, uncomment the \texttt{draftfalse}
  command in this document.}

In some instances internal languages of toposes allow mathematicians to reason
\emph{synthetically} about mathematical structures in a more concise and natural way. 
While Homotopy Type Theory provides an internal language for $(\infty,1)$-toposes, 
it is also possible to consider type theories that correspond to $(\infty,1)$-toposes with extra
structure of interest in algebraic and differential geometry.
The kind of extra structure relevant to this work is a \emph{differential cohesive} or \emph{cohesive} one.
A cohesive structure on a topos might be used to access topological structure on the objects 
and a differential cohesive structure admits using smooth structure.
For example, in real-cohesive type theory constructed by \citet{Shulman2015}, there is a type $\mathbb S^1$ for the topological circle 
and the usual homotopy type $S^1$ which is different from $\mathbb S^1$ but the two are connected by $\shape\mathbb S^1=S^1$ 
where $\shape$ is a part of the cohesive structure.

This work aims at constructing a differential cohesive homotopy type theory, 
which would allow synthetic reasoning not just about ordinary smooth manifolds,
but also $\infty$-stack variants which are of great interest in current pure mathematics.
For example, in \citep{SatiSchreiberStasheff2012} spaces locally modeled on 2- and 6-types are important, 
which are supported by and may already be reasoned about in a fragment of 
differential cohesive homotopy type theory used in \cite{Wellen2017} to develop the basics of Cartan geometry.

The most advanced type theory presented in this work supports all operations of differential cohesion 
arranged in an easy to work with pattern,
but lacking dependent types and identity types.


\begin{wrapfigure}{R}{0.5\textwidth}
\begin{center}
  \begin{tikzpicture}[node distance=2cm]
    \node[color=blue] (R) {$\Red$};
    \node[right of=R,color=blue] (I) {$\Cored$};
    \node[right of=I,color=blue] (E) {$\Wat$};
    \node[below of=I, node distance=1.6cm] (S) {$\shape$};
    \node[below of=E, node distance=1.6cm] (b) {$\flat$};
    \node[right of=b] (s) {$\sharp$};
  
    \path (R) to node[color=blue] {$\dashv$} (I);
    \path (I) to node[color=blue] {$\dashv$} (E);
    \path (S) to node {$\dashv$} (b);
    \path (b) to node {$\dashv$} (s);

    \path (I) to node[color=blue] {$\submodality$} (S);
    \path (E) to node[color=blue] {$\submodality$} (b);
  \end{tikzpicture}
\end{center}
\caption{On the bottom, the real-cohesion operations $\shape$ (the connected
  components with discrete topology on 1-toposes and the fundamental $\infty$-groupoid in the $(\infty,1)$-topos version), 
  $\flat$ (the underlying set with discrete
  topology), and $\sharp$ (the underlying set with codiscrete topology). On the
  top, the differential-cohesion operations $\Red$ (the underlying space without infinitesimal directions), 
  $\Cored$ (a space with the same points but all maps from it to any other space induce trivial maps on tangent spaces), and
  $\Wat$ (on manifolds, this is just the discrete underlying set; applied to the right stacks, it might be a coefficient object for interesting cohomology theories). The inclusion symbol $F \subset G$ indicates that the image of
  $F$ is contained in the image of $G$.}
\label{fig:modalities}
\end{wrapfigure}


\paragraph{Cohesion in adjoint type theory.}

In previous work, \citet{Shulman2015} constructs a variant of homotopy type
theory for $(\infty,1)$-toposes with an additional ``cohesive'' structure. In
this setting, derivations in the type theory correspond to continuous maps that
respect the cohesive structure of the space, providing a framework for a wide
variety of synthetic proofs. For example, Brower's fixed-point theorem, which
states that all continuous maps over the topological disk have a fixed point has
no synthetic proof in ordinary homotopy type theory, but has a proof in cohesive
HoTT~\citep{Shulman2015}.


To capture cohesion in the type theory, \citet{Shulman2015} uses a \emph{modal}
type theory~\citep{Pfenning2001} to describe the categorical structure of
cohesive $(\infty,1)$-toposes. This structure consists of three
\emph{modalities} \tocite{modalities} that form an adjoint triple
$\shape \dashv \flat \dashv \sharp$; they are described in
\cref{fig:modalities}. In addition, variables in the type theory are marked as
either \emph{cohesive} or \emph{crisp} depending on how they are used in a
term---a typing judgment $\Gamma \mid \Delta \vdash e : \tau$ is continuous on
the cohesive variables in $\Delta$, but may be discontinuous on the crisp
variables in $\Gamma$. The modalities $\flat$ and $\sharp$ are defined as type
and term formation and introduction rules, while $\shape$ is defined as a higher
inductive type.

In particular, \citeauthor{Shulman2015} defines the $\shape$ functor as a
localization using the Dedekind-reals. While this approach admits very
useful constructions of familiar topological spaces using the real numbers, like
the \emph{topological spheres} or disks which have the right homotopy type, it
is also interesting to look at general cohesion, where $\shape$ is not a priori
linked to the Dedekind-reals.


\paragraph{Differential cohesion in adjoint type theory.}

In this work we aim to construct a type theory similar to Shulman's cohesive
type theory that captures the additional structure of what are called
differential cohesive toposes. \FW{I am quite sure, that Mike invented the term
  ``real cohesion'' for his more specific version of cohesion, where $\shape$ is
  defined as localization using the Dedekind-reals. So I removed ``real'' on
  some instances.} Differential cohesive toposes, used by \citet{Schreiber2013}
to reason about spaces with geometric structures of interest to modern physics,
have already been explored from within a type theory by \citet{Wellen2017}.
There, just a small fragment of differential cohesive type theory is used to
define and use notions of differential geometry. 
The only difference to homotopy type theory
 is the existence of the $\Cored$ modality given by axioms.
The comonadic modalities, $\Red$, $\Wat$, $\flat$ can not be added to a type theory in this way -- 
the rules have to be changed.
And it is sometimes better, to also implement the monadic modalities by modifying the rules instead of adding axioms.

A full type theory for differential cohesion would cover not just the three
differential modalities, but also the three cohesive modalities, all shown
together in \cref{fig:modalities}. Unfortunately, Shulman's construction cannot
be extended directly from real to differential cohesion---in typical models of
differential cohesion the Dedekind reals do not correspond to the real line as a
smooth space! Furthermore, it is not known if there is a way to define the
differentially cohesive real line internally to the type theory. 

Following the \citeyear{Licata2016} paper on cohesion, \citet{Licata2017}
developed a general construction for non-dependent modal type theories, which we
instantiate for differential cohesion in this paper. This takes place in two
stages.

First, in the absence of dependent types, we cannot utilize Shulman's definition
of $\shape$ as a higher inductive type characterized by its localization
properties. Instead we give introduction and elimination rules for $\shape$
internally to the type theory, guided by \citeauthor{Licata2017}'s general
construction. The result is a typing judgment with three sorts of contexts,
written $\Gamma \mid \Delta \mid \Xi \vdash e : \tau$, where $\Gamma$ and
$\Delta$ still hold crisp and cohesive variables, respectively, and where $\Xi$
contains \emph{shapely} variables, which are constant on the connected
components of the topological structure. In particular, crisp variables
correspond directly to the modality $\flat$, while shapely variables correspond
to the modality $\shape$.

Next, we extend the judgment to differentially cohesive variables. The judgment
$\Gamma \mid \Delta \mid \Theta \mid \Lambda \mid \Xi \vdash e : \tau$ uses
variables in the following way:
\begin{center} \begin{tabular}{lll}
    $\Gamma$ & crisp  &  $\flat$ \\
    $\Delta$ & reduced & $\Red$ \\
    $\Theta$ & differentially cohesive & \\
    $\Lambda$ & coreduced & $\Cored$ \\
    $\Xi$ & shapely & $\shape$
\end{tabular} \end{center}
%
We add the modalities $\Red$, $\Cored$, and $\Wat$ to this judgment as inference
rules using \citeauthor{Licata2017}'s framework. Because the type theory is
derived straightforwardly from this framework, we get some meta-theoretic
results---substitution, structural rules, \etc---for free.\footnote{We must
  still be careful about the proof that our type system corresponds exactly to
  an instance of \citeauthor{Licata2017}'s framework.} \JP{Do we want to give
  any further details of the mode theory here?}

\todo{Key question: what are the contributions of this work?}

Future work will extend this type theory to dependent and identity types.
\todo{Indicate what is hard about this.}


\todo{All--please edit bibtex entries if necessary.}

% Using a bibstyle that is compatible with author/year citations and also the
% eprint field (for arXiv)
\bibliographystyle{hplain}
\bibliography{bibliography}

\end{document}