\documentclass{article}
%\usepackage[lmargin=2.5cm,rmargin=2.5cm,tmargin=2.5cm,bmargin=2.5cm]{geometry}
\usepackage{fullpage}
\usepackage{amsmath}
\usepackage{amssymb}
\usepackage{amsthm}
\usepackage{phonetic} % for esh
\usepackage{enumerate}

\usepackage{tikz}
\usetikzlibrary{arrows}

\usepackage{authblk} % For authors' affiliations
\makeatletter
\renewcommand\AB@affilsepx{, \protect\Affilfont}
\makeatother


\usepackage[hidelinks]{hyperref}
\usepackage{natbib}

\usepackage[capitalize,noabbrev]{cleveref} % must be loaded after hyperref

\usepackage{xspace}

\usepackage{wrapfig}

% Modalities
\newcommand{\Red}{\Re}
\newcommand{\Cored}{\Im}
\newcommand{\Wat}{\&}
\newcommand{\shape}{\ensuremath{\mathord{\raisebox{0.5pt}{\text{\rm\esh}}}}}
\newcommand{\submodality}{\rotatebox[origin=c]{90}{$\subset$}}


% Latin  Abbr
\newcommand{\etal}{\emph{et al.}\xspace}
\newcommand{\eg}{\emph{e.g.,}\xspace}
\newcommand{\ie}{\emph{i.e.,}\xspace}
\newcommand{\etc}{\emph{etc.}\xspace}

%%%%%%%%%%%%%%
%% Comments %%
%%%%%%%%%%%%%%
\usepackage{draft}

\newnote[Jacob]{JAG}{blue}
\newnote[Max]{MN}{blue}
\newnote[Jennifer]{JP}{blue}
\newnote[Mitchell]{MMR}{blue}
\newnote[Felix]{FW}{blue}
\newnote[Dan]{DL}{blue}
\newnote[Mike]{MS}{blue}

\newnote{fixme}{red}
\newnote{note}{red}
\newnote{todo}{red}
\newnote[Cite:]{tocite}{red}




% use \draftfalse to turn off comments
% use \drafttrue to turn on comments (default)
%\draftfalse

\title{Differential Cohesive Type Theory (Extended Abstract)\thanks{This
    material is based upon work supported by the National Science Foundation
    under Grant Number DMS 1321794.}}
\author{Jacob A.\,Gross}
\affil[1]{University of Pittsburgh}
\author[2]{Max S.\, New}
\affil[2]{Northeastern University}
\author[3]{Jennifer Paykin}
\affil[3]{University of Pennsylvania}
\author[4]{Mitchell Riley}
\affil[4]{Wesleyan University}
\author[5]{Felix Wellen}
\affil[5]{Karlsruhe Institute of Technology}
\author[4]{Daniel R.\,Licata}
\author[6]{Michael Shulman}
\affil[6]{University of San Diego}
\date{}

%\author{Jacob A.\,Gross \and Max S.\, New \and Jennifer Paykin \and Mitchell Riley
%  \and Felix Wellen \and Daniel R.\,Licata \and Michael Shulman}

\begin{document}
\maketitle

\JP{How to use comments: use your initials as a command to make a comment
  terminated by your name, \eg \texttt{\textbackslash JP\{my comments\}}. May
  also use \texttt{fixme}, \texttt{note}, \texttt{todo}, and \texttt{tocite}.
  You can change your keyword, color, etc in the preamble. To remove all
  comments, \eg when submitting, uncomment the \texttt{draftfalse}
  command in this document.}

The internal language of a type theory allows mathematicians to reason
\emph{synthetically} about mathematical structures. While Homotopy Type Theory
provides an internal language for $(\infty,1)$-toposes, it is also possible to
consider type theories that correspond to other structures. In previous work,
\citet{Shulman2015} and \citet{Licata2016} construct a type theory for
$(\infty,1)$-toposes with an additional ``cohesive'' structure mirroring cohesion of
topological spaces. In this setting, derivations in the type theory correspond
to continuous maps that respect the cohesive structure of the space, which
allows for a synthetic proof of, for example, Brower's fixed-point theorem,
which states that all continuous maps over the topological disk have a fixed
point~\citep{Shulman2015}. This theorem has no synthetic proof in ordinary
homotopy type theory.

To capture cohesion in the type theory, \citep{Shulman2015} and
\citep{Licata2016} use \emph{modal} type theories~\tocite{pfenning and davis?}
to describe the categorical structure of cohesive $(\infty,1)$-toposes, which is
shown in black in \cref{fig:modalities}. \todo{say more about the type theory,
  difference between the two papers.}

\begin{figure}
\begin{center}
  \begin{tikzpicture}[node distance=2cm]
    \node (R) {$\Red$};
    \node[right of=R] (I) {$\Cored$};
    \node[right of=I] (E) {$\Wat$};
    \node[below of=I, node distance=1.6cm] (S) {$\shape$};
    \node[below of=E, node distance=1.6cm] (b) {$\flat$};
    \node[right of=b] (s) {$\sharp$};
  
    \path (R) to node {$\dashv$} (I);
    \path (I) to node {$\dashv$} (E);
    \path (S) to node {$\dashv$} (b);
    \path (b) to node {$\dashv$} (s);

    \path (I) to node {$\submodality$} (S);
    \path (E) to node {$\submodality$} (b);
  \end{tikzpicture}
\end{center}
\caption{On the bottom, the real-cohesion operations $\shape$ (the connected
  components with discrete topology), $\flat$ (the underlying set with discrete topology), and $\sharp$ (the underlying set with
  codiscrete topology). On the top, the differential-cohesion operations $\Red$
  (the reduced topology), $\Cored$ (the coreduced topology), and $\Wat$ (??).
  \todo{use colors to distinguish real-cohesive vs differential-cohesive
    modalities. Also include the quadruple adjoint functors between different
    categories of spaces.} The inclusion symbols in the diagram indicate an
  inclusion of two of the four subtoposes these functors reflect into. }
\label{fig:modalities}
\end{figure}

In this work we extend Shulman's real-cohesive type theory to reason not just
about the cohesive structure of topological spaces, but also the differential
structure. This additional structure, explored by \citet{Schreiber2013}, is
called \emph{differential cohesion}, and is useful for applications in physics
and Cartan geometry~\citep{Wellen2017}. In particular, \citeauthor{Wellen2017}
considers a small fragment of differential cohesive type theory that uses just
the $\Cored$ modality. A full type theory for differential cohesion would cover
not just the three differential modalities, but also the three real-cohesive
modalities, all shown together in \cref{fig:modalities}.

Following the \citeyear{Licata2016} paper on real cohesion, \citet{Licata2017}
develop a general construction for non-dependent modal type theories, which we
instantiate for differential cohesion in this paper. Future work will extend
this type theory to dependent and identity types.

\section{Real cohesion in adjoint type theory}

\todo{start with real cohesion}

\section{Differential cohesion in adjoint type theory}

\todo{extend to differential cohesion}

\todo{All--please edit bibtex entries if necessary.}

% Using a bibstyle that is compatible with author/year citations and also the
% eprint field (for arXiv)
\bibliographystyle{hplain}
\bibliography{bibliography}

\end{document}